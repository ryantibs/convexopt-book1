\chapter{Bregman divergences}

\section{Definition and properties}

% In what ways do these preserve properties of distances? In what ways do they
% not? 

\section{Bregman projections and proximals}

% In what ways do these preserve properties of projections/proximals? In what
% ways do they not?

% This paper: https://cmps-people.ok.ubc.ca/bauschke/Research/103.pdf may help
% as a resource 

\section{Conjugacy connections}

% For any Legendre function h and its conjugate h^*:
% D_h(y, x) = D_{h^*}(u, v)
% with u = \nabla h(x) and v = \nabla h(y). Note the reversal of the order of
% arguments! Proof is elementary

\section{Exponential families*}

% In exponential families, the gradient of the LPF is the map between natural
% parameter and the mean (cf Barndoff and Nielsen, Brad, Martin and Mike).  

% Maybe these notes help as as resource (aside from those books):
% https://sites.stat.washington.edu/mmp/courses/stat538/winter12/Handouts/l7-exponential.pdf 

% The inverse, the map from the mean to the natural parameter, is the gradient
% of the conjugate! 

% Deviance in exponential family can be written in terms of a Bregman
% divergence, in TWO ways. These two ways are related by conjugacy.

% Does this help for optimism / covariance formula type results? I think Brad's
% work shows that it does? 

\section{Proper loss functions*}

% Connection between proper losses and conjugacy (cf Savage, Gneiting and
% Raftery, etc.) 

% Conjugacy relation gives rise to a way to define a surrogate loss (credit to
% John Duchi)