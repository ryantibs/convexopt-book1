\chapter{Fine properties of the lasso}
\label{chap:lasso}

\section{Structure of solutions}
\label{sec:lasso_structure}

% give basic properties (unique fit, unique \ell_1 norm)
% give algebraic form of lasso solutions by KKT manipulation 

\section{Geometry of solutions}

% give the primal-dual picture as in Tibshirani and Taylor, or Ali and
% Tibshirani; interpret, and re-hash the implications from Example 10.6. 

\section{Uniqueness of solutions}

\section{Degrees of freedom}

\section{Screening rules*}

\section{Homotopy algorithm*}

\SkipTocEntry\section*{Chapter notes}

% much of this carries over to the generalized lasso

% much more can be said in terms of statistical properties, we have focused
% mainly on deterministic (not probabilistic) properties which are derivable
% from convex analytic  
