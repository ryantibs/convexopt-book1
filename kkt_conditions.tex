\chapter{Karush-Kuhn-Tucker conditions}
\label{chap:kkt_conditions}

\section{Saddle point condition}
\label{sec:saddle_point_condition}

In this chapter, we will explore conditions which characterize solutions in
constrained optimization problems, the most well-known being a set of conditions
known as the Karush-Kuhn-Tucker (KKT) conditions. This flows naturally from our 
study of Lagrange duality in the last chapter. In fact, the last chapter already
established a critical relationship between primal and dual solutions and what
are known as saddle points of the Lagrangian. These arguments were given in 
passing, and here we revisit them, as they will lay the foundation for the KKT 
conditions, which we cover next.           

As in the last chapter, consider a primal problem
\begin{equation}
\label{eq:primal_problem2}
\begin{alignedat}{2}
&\minimize_x \quad && f(x) \\
&\st \quad && h_i(x) \leq 0, \; i=1,\dots,m \\ 
& && \ell_j(x) = 0, \; j=1,\dots,k,
\end{alignedat}
\end{equation}
whose associated Lagrangian and dual function are
\begin{align*}
L(x,u,v) &= f(x) + \sum_{i=1}^m u_i h_i(x) + \sum_{j=1}^k v_j \ell_j(x), \\ 
g(u,v) &= \inf_x \, L(x,u,v), 
\end{align*}
and whose associated dual problem is 
\begin{equation}
\label{eq:dual_problem2}
\maximize_{u,v} \quad g(u,v) \quad \st \quad u \geq 0.
\end{equation}
We do not assume that \eqref{eq:primal_problem2} is convex, though recall, the 
dual problem \eqref{eq:dual_problem2} is always convex. In general, we say that
a triplet \smash{$\bar{x}, \bar{u}, \bar{v}$} is a \emph{saddle point} of the
Lagrangian if  
\index{Lagrangian function!saddle point} 
\begin{equation}
\label{eq:lagrangian_saddle_point2}
L(\bar{x}, u, v) \leq L(\bar{x}, \bar{u}, \bar{v}) \leq L(x, \bar{u}, \bar{v}),
\quad \text{for any $x$ and any $u \geq 0, \, v$}.
\end{equation}
In other words, starting at \smash{$\bar{x}, \bar{u}, \bar{v}$}, the condition
says: if we move the primal variable away from \smash{$\bar{x}$} then $L$ can
only increase; if we move the dual variables away from \smash{$\bar{u},
  \bar{v}$}, then $L$ can only decrease.    

The following is a useful characterization of primal and dual solutions via
saddle points.  

\begin{Theorem}
\label{thm:saddle_point_condition}
For any primal problem \eqref{eq:primal_problem2} and its dual
\eqref{eq:dual_problem2}, and any primal feasible \smash{$\bar{x}$} and dual
feasible \smash{$\bar{u}, \bar{v}$}, the triplet \smash{$\bar{x}, \bar{u},
  \bar{v}$} is a saddle point of the Lagrangian
\eqref{eq:lagrangian_saddle_point2} if and only if \smash{$\bar{x}$} is primal
optimal, \smash{$\bar{u}, \bar{v}$} are dual optimal, and strong duality holds.
\end{Theorem}

Though the arguments behind this result were already given in Chapter 
\ref{chap:lagrangian_duality}, it will be helpful to consolidate them here.
First, note that the saddle point condition
\eqref{eq:lagrangian_saddle_point2} has a couple of equivalent forms:
\begin{equation}
\label{eq:lagrangian_saddle_point3}
\sup_{u \geq 0, \, v} \, L(\bar{x}, u, v) = L(\bar{x}, \bar{u}, \bar{v}) =
\inf_x \, L(x, \bar{u}, \bar{v}), 
\end{equation}
as well as
\begin{equation}
\label{eq:lagrangian_saddle_point4}
f(\bar{x}) = L(\bar{x}, \bar{u}, \bar{v}) = g(\bar{u}, \bar{v}).
\end{equation}
The right-hand side in the above display is due to the definition of the dual
function $g$, whereas the left-hand side uses the representation of $f$ as a
supremum of the Lagrangian, from Property \parref{par:max_min}. To prove    
Theorem \ref{thm:saddle_point_condition} we simply observe that
\eqref{eq:lagrangian_saddle_point4} is equivalent to \smash{$\bar{x}, \bar{u}, 
  \bar{v}$} being primal and dual solutions with zero duality gap. 

\section{Karush-Kuhn-Tucker conditions}
\label{sec:kkt_conditions}

We say that a triplet \smash{$\bar{x}, \bar{u}, \bar{v}$} is a
\emph{Karush-Kuhn-Tucker (KKT) point} associated with the primal and 
dual pair \eqref{eq:primal_problem2} and \eqref{eq:dual_problem2} if it
satisfies the following conditions: 
\index{KKT conditions}
\begin{alignat}{2}
\label{eq:kkt_stationarity}
&0 \in \partial_x L(\bar{x}, \bar{u}, \bar{v}) \quad
&& \text{(stationarity)} \\
\label{eq:kkt_complementary_slackness} 
&\bar{u}_i h_i(\bar{x}) = 0, \; i = 1,\dots,m \quad 
&& \text{(complementary slackness)} \\
\label{eq:kkt_primal_feasibility}
&h_i(\bar{x}) \leq 0, \; i = 1,\dots,m, \; \text{and} \; 
\ell_j(\bar{x}) = 0, \; j = 1,\dots,k \quad 
&& \text{(primal feasibility)} \\ 
\label{eq:kkt_dual_feasibility}
&\bar{u}_i \geq 0, \; i = 1,\dots,m. \quad
&& \text{(dual feasibility)}
\end{alignat}
To be clear, in \eqref{eq:kkt_stationarity}, called the stationarity condition,
the subdifferential is being computed with respect to the $x$ component of the
Lagrangian. 

The next result shows that the KKT conditions offer yet another equivalent
characterization of saddle point condition (together with feasibility).  

\begin{Lemma}
\label{lem:saddle_point_kkt}
For any primal feasible \smash{$\bar{x}$} and dual feasible \smash{$\bar{u}, 
  \bar{v}$}, the saddle point condition \eqref{eq:lagrangian_saddle_point2} is
equivalent to stationarity and complementary slackness,
\eqref{eq:kkt_stationarity} and \eqref{eq:kkt_complementary_slackness}.
\end{Lemma}

The proof is straightforward, if we use the equivalent form of the saddle point
condition \eqref{eq:lagrangian_saddle_point4}. The second equality in
\eqref{eq:lagrangian_saddle_point4} is equivalent to the fact that
\smash{$\bar{x}$} minimizes \smash{$L(\cdot, \bar{u}, \bar{v})$}, which is
equivalent to stationarity \eqref{eq:kkt_stationarity}. The first equality in
\eqref{eq:lagrangian_saddle_point4} is implied by complementary slackness
\eqref{eq:kkt_complementary_slackness}:
\[
L(\bar{x}, \bar{u}, \bar{v}) = f(\bar{x}) + 
\sum_{i=1}^m \underbrace{\bar{u}_i h_i(\bar{x})}_{=\,0} + 
\sum_{j=1}^k \underbrace{\bar{v}_j \ell_j(\bar{x})}_{=\,0} 
= f(\bar{x}),
\]
where we have used primal feasibility. Furthermore, the first equality in
\eqref{eq:lagrangian_saddle_point4} also implies complementary slackness: if
\smash{$f(\bar{x}) = L(\bar{x}, \bar{u}, \bar{v})$}, then (again using primal
and dual feasibility) we conclude that all summands must be zero in the middle
expression of the last display, which leads to
\eqref{eq:kkt_complementary_slackness}. This completes the proof the theorem.

Combining Theorem \ref{thm:saddle_point_condition} and Lemma
\ref{lem:saddle_point_kkt} yields the following important result. 

\index{KKT conditions}
\begin{Theorem}
\label{thm:kkt_conditions}
For any primal problem \eqref{eq:primal_problem2} and its dual
\eqref{eq:dual_problem2}, a triplet \smash{$\bar{x}, \bar{u}, \bar{v}$}
satisfies the KKT conditions
\eqref{eq:kkt_stationarity}--\eqref{eq:kkt_dual_feasibility} if and only if 
\smash{$\bar{x}$} is primal optimal, \smash{$\bar{u}, \bar{v}$} are dual
optimal, and strong duality holds. 
In other words, the KKT conditions are always sufficient for optimality, and
necessary under strong duality.  
\end{Theorem}

examples

quad with eq constraint

lasso??

\index{lasso!optimality conditions}

SVM

\index{svm!optimality conditions}

\subsection{The role of convexity*}

what goes here? is this where we talk about splitting up subdifferentials?

\section{Primal solutions from dual solutions}

\section{Constraint qualification*}

should we cover constraint qualification? 
somehow try to cover lagrange multipliers as a consquence?

\section{Equivalence of constrained and penalized forms*}

\section{Plurality of dual problems*}
\label{sec:plurality_dual_problems}

?? maybe move this to being an exercise or several exercises

\begin{xcb}{Exercises}
\begin{enumerate}[label=\thechapter.\arabic*]
\settowidth{\leftmargini}{00.00.\hskip\labelsep}
\item \label{ex:lasso_dantzig}
Compare these two? And how about basis pursuit?

\item \label{ex:simplex_projection}

\end{enumerate}
\end{xcb}