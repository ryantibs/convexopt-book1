\chapter{Karush-Kuhn-Tucker conditions}
\label{chap:kkt_conditions}

\section{Saddle point condition}
\label{sec:saddle_point_conditions}

In this chapter, we will explore conditions which characterize solutions in
constrained optimization problems, the most well-known being a set of conditions
known as the Karush-Kuhn-Tucker (KKT) conditions. This flows naturally from our 
study of Lagrange duality in the last chapter. In fact, the last chapter already
established a key relationship between primal and dual solutions and what are
called saddle points of the Lagrangian, which we revisit here. This will then
lay the foundation for the KKT conditions, which we cover next.   

As in the last chapter, consider a primal problem
\begin{equation}
\label{eq:primal_problem2}
\begin{alignedat}{2}
&\minimize_x \quad && f(x) \\
&\st \quad && h_i(x) \leq 0, \; i=1,\dots,m \\ 
& && \ell_j(x) = 0, \; j=1,\dots,k,
\end{alignedat}
\end{equation}
whose associated Lagrangian and dual function are
\begin{align*}
L(x,u,v) &= f(x) + \sum_{i=1}^m u_i h_i(x) + \sum_{j=1}^k v_j \ell_j(x), \\ 
g(u,v) &= \inf_x \, L(x,u,v), 
\end{align*}
and whose associated dual problem is 
\begin{equation}
\label{eq:dual_problem2}
\maximize_{u,v} \quad g(u,v) \quad \st \quad u \geq 0.
\end{equation}
We do not assume that \eqref{eq:primal_problem2} is convex, though recall, the 
dual problem \eqref{eq:dual_problem2} is always convex. In general, we say that
a triplet \smash{$\bar{x}, \bar{u}, \bar{v}$} is a \emph{saddle point} of the
Lagrangian if  
\index{Lagrangian function!saddle point} 
\begin{equation}
\label{eq:lagrangian_saddle_point2}
L(\bar{x}, u, v) \leq L(\bar{x}, \bar{u}, \bar{v}) \leq L(x, \bar{u}, \bar{v}),
\quad \text{for any $x$ and any $u \geq 0, \, v$}.
\end{equation}
In other words, starting at \smash{$\bar{x}, \bar{u}, \bar{v}$}, this condition
says: if we move the primal variable away from \smash{$\bar{x}$} then $L$ can
only increase, and furthermore, if we move the dual variables away from
\smash{$\bar{u}, \bar{v}$}, then $L$ can only  decrease.  The following is a
simple but useful characterization of primal and dual solutions via saddle
points. 

\begin{Theorem}
\label{thm:saddle_point_condition}
For any primal problem \eqref{eq:primal_problem2} and its dual
\eqref{eq:dual_problem2}, and any primal feasible \smash{$\bar{x}$} and dual  
feasible \smash{$\bar{u}, \bar{v}$}, the following holds: \smash{$\bar{x},
  \bar{u}, \bar{v}$} is a saddle point of the Lagrangian if and only if
\smash{$\bar{x}$} is primal optimal, \smash{$\bar{u}, \bar{v}$} are dual
optimal, and strong duality holds. 
\end{Theorem}

Though the arguments behind this result were already given in Chapter 
\ref{chap:lagrangian_duality}, it will be helpful to consolidate them here.
First, note that the saddle point condition
\eqref{eq:lagrangian_saddle_point2} has a couple of equivalent forms:
\begin{equation}
\label{eq:lagrangian_saddle_point3}
\sup_{u \geq 0, \, v} \, L(\bar{x}, u, v) = L(\bar{x}, \bar{u}, \bar{v}) =
\inf_x \, L(x, \bar{u}, \bar{v}), 
\end{equation}
as well as
\begin{equation}
\label{eq:lagrangian_saddle_point4}
f(\bar{x}) = L(\bar{x}, \bar{u}, \bar{v}) = g(\bar{u}, \bar{v}).
\end{equation}
The right-hand side in the above display is due to the definition of the dual
function $g$, whereas the left-hand side uses the representation of $f$ as a
supremum of the Lagrangian, from Property \parref{par:max_min}. To prove    
Theorem \ref{thm:saddle_point_condition} we simply observe that
\eqref{eq:lagrangian_saddle_point4} is equivalent to \smash{$\bar{x}, \bar{u}, 
  \bar{v}$} being primal and dual solutions with zero duality gap. 

\section{Karush-Kuhn-Tucker conditions}
\label{sec:kkt_conditions}

 This will lay the
foundation for the the KKT conditions, which will be covered next. 

examples

quad with eq constraint

lasso??

\index{lasso!optimality conditions}

SVM

\index{svm!optimality conditions}

\subsection{The role of convexity*}

what goes here? is this where we talk about splitting up subdifferentials?

\section{Primal solutions from dual solutions}

\section{Constraint qualification*}

should we cover constraint qualification? 
somehow try to cover lagrange multipliers as a consquence?

\section{Equivalence of constrained and penalized forms*}

\section{Plurality of dual problems*}
\label{sec:plurality_dual_problems}

?? maybe move this to being an exercise or several exercises

\begin{xcb}{Exercises}
\begin{enumerate}[label=\thechapter.\arabic*]
\settowidth{\leftmargini}{00.00.\hskip\labelsep}
\item \label{ex:lasso_dantzig}
Compare these two? And how about basis pursuit?

\item \label{ex:simplex_projection}

\end{enumerate}
\end{xcb}