\chapter{Duality}
\label{chap:duality}

\section{LP duality*}
\label{sec:lp_duality}

Duality is fascinating topic in mathematical optimization: the basic arguments
used in the theory of duality are elementary, and yet they can lead to powerful
and far-reaching conclusions.    

The story begins, for us, with linear programs (LPs). Consider a generic LP of
the form 
\begin{equation}
\label{eq:lp_primal}
\begin{alignedat}{2}
&\minimize_x \quad && c^\T x \\
&\st && Ax \leq b \\
& && Gx = h, 
\end{alignedat}
\end{equation}
where $c \in \R^d$, $A \in \R^{m \times d}$, $b \in \R^m$, $G \in \R^{k \times 
  d}$, $h \in \R^k$. The reason we start the chapter by studying LPs is that,   
in this problem class, we can build up dual problems ``constructively''; this 
constructive approach is not possible for general optimization problems, and
helps us appreciate the importance and elegance of Lagrange duality, which is
covered next.     

The fundamental question that underlies the study of duality is as follows: what
is the tightest lower bound we can form on the optimal criterion value $f^\star
= c^\T x^\star$ in \eqref{eq:lp_primal}? To address this, suppose that $u \in
\R^m$ and $v \in \R^k$ are arbitrary vectors---which we call \emph{dual 
  variables} in this context---with $u$ nonnegative in each component, $u \geq   
0$. Provided that $x \in \R^d$ is feasible for problem \eqref{eq:lp_primal}, it
holds that $u^\T (Ax - b) \leq 0$ and $v^\T (Gx - h) = 0$, and so, adding these
together gives        
\begin{equation}
\label{eq:lp_dual_nonnegativity}
u^\T (Ax - b) + v^\T (Gx - h) \leq 0.
\end{equation}
In order to obtain a lower bound on the criterion value $c^\T x$, we rearrange
the above into
\[
(-A^\T u - G^\T v)^\T x \geq -b^\T u - h^\T v.
\]
The key observation is that this provides the lower bound we desire provided
that the dual variables $u,v$ are chosen such that $-A^\T u - G^\T v= c$. This
is true for any feasible $x$, thus taking an infimum over all such $x$ gives   
\[
f^\star \geq -b^\T u - h^\T v, \quad \text{for any $u,v$ such that $-A^\T u -
  G^\T v = c$ and $u \geq 0$}.
\]
Finally, yo make this lower bound as tight as possible, we maximize the
right-hand side above,  
\begin{equation}
\label{eq:lp_weak_duality}
f^\star \geq \, \underbrace{\sup \big\{ -b^\T u - h^\T v :  A^\T u + G^\T v =
  -c, \, u \geq 0 \big\}}_{g^\star}. 
\end{equation}
Now, the right-hand side in \eqref{eq:lp_weak_duality}, which we may denote as 
$g^\star = -b^\T u^\star - h^\T v^\star$, is itself the optimal criterion value
associated with an optimization problem, indeed an LP,     
\index{dual problem!linear program}
\begin{equation}
\label{eq:lp_dual}
\begin{alignedat}{2}
&\maximize_{u,v} \quad && b^\T u - h^\T v \\
&\st && A^\T u + G^\T v = -c \\
& && u \geq 0.
\end{alignedat}
\end{equation}
In this context, we call \eqref{eq:lp_dual} the \emph{dual LP} and the original 
problem \eqref{eq:lp_primal} the \emph{primal LP}. Note that, by construction 
\eqref{eq:lp_weak_duality}, we have $f^\star \geq g^\star$: the optimal
criterion in the dual problem is a lower bound on the optimal criterion in the  
primal problem.  

There are several natural follow-up questions that we can ask: for example, when
does equality hold in \eqref{eq:lp_weak_duality}, $f^\star = g^\star$? And how
can we relate solutions $x^\star$ and $u^\star, v^\star$ in \eqref{eq:lp_primal}
and \eqref{eq:lp_dual}, respectively (beyond the optimal criterion values)? We
will address these questions, and more, over this chapter and the next
one. First, however, we must develop a theory of duality beyond LPs.

The attempt to move beyond linear programs exposes a shortcoming of the above 
approach: if the criterion $f$ is not linear, but the constraint functions are
all linear, then we will have no general way of manipulating the constraints---a
set of linear equalities and inequalities in $x$---to form a lower bound on
$f(x)$. But there is another way: Lagrange duality, as we will see next, applies
not only to LPs but to optimization problems broadly (even nonconvex ones).  

\section{Lagrangian duality}
\label{sec:lagrangian_duality}

Lagrangian duality (or Lagrange duality) starts with the same motivating
question as above, in the last section, but cast in a more general setting,
where we seek a lower bound on the optimal criterion value $f^\star$ in      
\begin{equation}
\label{eq:primal}
\begin{alignedat}{2}
&\minimize_x \quad && f(x) \\
&\st \quad && h_i(x) \leq 0, \; i=1,\ldots,m \\ 
& && \ell_j(x) = 0, \; j=1,\ldots,k.
\end{alignedat}
\end{equation}
At the moment, we do not assume that criterion $f$ or the constraint functions
$h_i$, $i=1,\dots,m$, and $\ell_j$, $j=1,\dots,k$ are convex. Thus, to be clear,
we do not assume that \eqref{eq:primal} is a convex problem. 

As before, let $u \in \R^m$ and $v \in \R^k$ be arbitrary vectors, with $u \geq
0$, which we call \emph{dual variables} in the current context. Using the same
basic idea as in \eqref{eq:lp_dual_nonnegativity}, we now \emph{add} this nonpositive
quantity to the criterion to get a lower bound,   
\[
f(x) \geq \,\underbrace{f(x) + u^\T (Ax - b) + v^\T (Gx - h)}_{L(x,u,v)}.
\]   
The quantity on the right hand side above is called the \emph{Lagrangian} 
associated with \eqref{eq:primal}, evaluated at the primal $x$ and dual $u,v$
variables. 

\begin{Example}
LPs.
\end{Example}

\section{Interpretations}
\label{sec:duality_interpretations}

\section{SDP duality*}
\label{sec:sdp_duality}

% for SDP duality: take a look at 
% https://www.mit.edu/~parrilo/cdc03_workshop/ejc03_comp.pdf
% https://ocw.mit.edu/courses/electrical-engineering-and-computer-science/6-251j-introduction-to-mathematical-programming-fall-2009/readings/MIT6_251JF09_SDP.pdf