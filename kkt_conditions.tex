\chapter{Karush-Kuhn-Tucker conditions}
\label{chap:kkt_conditions}

start by (re)stating the saddle point theorem?

if x,u,v is a saddle point of the Lagrangian then x is primal optimal and u,v
are dual optimal

under strong duality, if x is primal optimal and u,v are dual optimal, then
x,u,v is a saddle point of the Lagrangian.

should we cover constraint qualification? 
somehow try to cover lagrange multipliers as a consquence?

\section{Primal solutions from dual solutions}


\begin{xcb}{Exercises}
\begin{enumerate}[label=\thechapter.\arabic*]
\settowidth{\leftmargini}{00.00.\hskip\labelsep}
\item \label{ex:lasso_dantzig}
Compare these two? And how about basis pursuit?

\item \label{ex:simplex_projection}

\end{enumerate}
\end{xcb}