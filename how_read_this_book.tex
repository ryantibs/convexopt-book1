\chapter{How To Read This Book}

\section{Notation and conventions}

% essentially all notation defined in passing so that the book reads more
% fluidly, but everything is also defined here for completeness...

% affine vs linear (linearly used loosly)

% analytical or closed-form solution: written in terms of functions and
% mathematical operations from a given generally-accepted set. (wikipedia defines
% the former to be broader than the latter, we use them synonomously)
% https://en.wikipedia.org/wiki/Closed-form_expression

% $\R^d$ denotes $d$-dimensional Euclidean space 
% all vectors are treated columns vectors, constructed as $x=(x_1,\ldots,x_d)$ 
% $\T$ is the transpose operator so $a^\T b$ is the standard inner product
% also write this as $\langle a, \rangle b$, etc etc

% $\one$ is used flexibly to denote 1s vector of whatever dimension

% $\geq$ is componentwise

% $1\{ ... \}$ indicator

% Notation: $x_1$ refers to the first component of $x$ or the first in a sequence
% of vectors, to be used flexibly 

% inf/sup written in two ways (sets versus operators)

% unless otherwise specified, sets are in euclidean space, functions act on
% euclidean space

% Definitions for MATRICES are not made separately from those over vector
% space. Interpret everything as a vector space. Optimization facts, subgradients,
% proximal operatotrs, etc. etc. Interpret l2 -> Froebnius, inner product, etc
% etc. 

% if statements are vacuous when conditions are vacuous, then ... be mature!
% e.g. we don't always explicitly rule out things like empty sets, functions
% that are identically infinite, etc. 

\section{Background level}

\section{Recommended paths}
